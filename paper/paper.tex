% Options for packages loaded elsewhere
% Options for packages loaded elsewhere
\PassOptionsToPackage{unicode}{hyperref}
\PassOptionsToPackage{hyphens}{url}
\PassOptionsToPackage{dvipsnames,svgnames,x11names}{xcolor}
%
\documentclass[
]{ltjsbook}
\usepackage{xcolor}
\usepackage{amsmath,amssymb}
\setcounter{secnumdepth}{-\maxdimen} % remove section numbering
\usepackage{iftex}
\ifPDFTeX
  \usepackage[T1]{fontenc}
  \usepackage[utf8]{inputenc}
  \usepackage{textcomp} % provide euro and other symbols
\else % if luatex or xetex
  \usepackage{unicode-math} % this also loads fontspec
  \defaultfontfeatures{Scale=MatchLowercase}
  \defaultfontfeatures[\rmfamily]{Ligatures=TeX,Scale=1}
\fi
\usepackage{lmodern}
\ifPDFTeX\else
  % xetex/luatex font selection
\fi
% Use upquote if available, for straight quotes in verbatim environments
\IfFileExists{upquote.sty}{\usepackage{upquote}}{}
\IfFileExists{microtype.sty}{% use microtype if available
  \usepackage[]{microtype}
  \UseMicrotypeSet[protrusion]{basicmath} % disable protrusion for tt fonts
}{}
\makeatletter
\@ifundefined{KOMAClassName}{% if non-KOMA class
  \IfFileExists{parskip.sty}{%
    \usepackage{parskip}
  }{% else
    \setlength{\parindent}{0pt}
    \setlength{\parskip}{6pt plus 2pt minus 1pt}}
}{% if KOMA class
  \KOMAoptions{parskip=half}}
\makeatother
% Make \paragraph and \subparagraph free-standing
\makeatletter
\ifx\paragraph\undefined\else
  \let\oldparagraph\paragraph
  \renewcommand{\paragraph}{
    \@ifstar
      \xxxParagraphStar
      \xxxParagraphNoStar
  }
  \newcommand{\xxxParagraphStar}[1]{\oldparagraph*{#1}\mbox{}}
  \newcommand{\xxxParagraphNoStar}[1]{\oldparagraph{#1}\mbox{}}
\fi
\ifx\subparagraph\undefined\else
  \let\oldsubparagraph\subparagraph
  \renewcommand{\subparagraph}{
    \@ifstar
      \xxxSubParagraphStar
      \xxxSubParagraphNoStar
  }
  \newcommand{\xxxSubParagraphStar}[1]{\oldsubparagraph*{#1}\mbox{}}
  \newcommand{\xxxSubParagraphNoStar}[1]{\oldsubparagraph{#1}\mbox{}}
\fi
\makeatother

\usepackage{color}
\usepackage{fancyvrb}
\newcommand{\VerbBar}{|}
\newcommand{\VERB}{\Verb[commandchars=\\\{\}]}
\DefineVerbatimEnvironment{Highlighting}{Verbatim}{commandchars=\\\{\}}
% Add ',fontsize=\small' for more characters per line
\usepackage{framed}
\definecolor{shadecolor}{RGB}{241,243,245}
\newenvironment{Shaded}{\begin{snugshade}}{\end{snugshade}}
\newcommand{\AlertTok}[1]{\textcolor[rgb]{0.68,0.00,0.00}{#1}}
\newcommand{\AnnotationTok}[1]{\textcolor[rgb]{0.37,0.37,0.37}{#1}}
\newcommand{\AttributeTok}[1]{\textcolor[rgb]{0.40,0.45,0.13}{#1}}
\newcommand{\BaseNTok}[1]{\textcolor[rgb]{0.68,0.00,0.00}{#1}}
\newcommand{\BuiltInTok}[1]{\textcolor[rgb]{0.00,0.23,0.31}{#1}}
\newcommand{\CharTok}[1]{\textcolor[rgb]{0.13,0.47,0.30}{#1}}
\newcommand{\CommentTok}[1]{\textcolor[rgb]{0.37,0.37,0.37}{#1}}
\newcommand{\CommentVarTok}[1]{\textcolor[rgb]{0.37,0.37,0.37}{\textit{#1}}}
\newcommand{\ConstantTok}[1]{\textcolor[rgb]{0.56,0.35,0.01}{#1}}
\newcommand{\ControlFlowTok}[1]{\textcolor[rgb]{0.00,0.23,0.31}{\textbf{#1}}}
\newcommand{\DataTypeTok}[1]{\textcolor[rgb]{0.68,0.00,0.00}{#1}}
\newcommand{\DecValTok}[1]{\textcolor[rgb]{0.68,0.00,0.00}{#1}}
\newcommand{\DocumentationTok}[1]{\textcolor[rgb]{0.37,0.37,0.37}{\textit{#1}}}
\newcommand{\ErrorTok}[1]{\textcolor[rgb]{0.68,0.00,0.00}{#1}}
\newcommand{\ExtensionTok}[1]{\textcolor[rgb]{0.00,0.23,0.31}{#1}}
\newcommand{\FloatTok}[1]{\textcolor[rgb]{0.68,0.00,0.00}{#1}}
\newcommand{\FunctionTok}[1]{\textcolor[rgb]{0.28,0.35,0.67}{#1}}
\newcommand{\ImportTok}[1]{\textcolor[rgb]{0.00,0.46,0.62}{#1}}
\newcommand{\InformationTok}[1]{\textcolor[rgb]{0.37,0.37,0.37}{#1}}
\newcommand{\KeywordTok}[1]{\textcolor[rgb]{0.00,0.23,0.31}{\textbf{#1}}}
\newcommand{\NormalTok}[1]{\textcolor[rgb]{0.00,0.23,0.31}{#1}}
\newcommand{\OperatorTok}[1]{\textcolor[rgb]{0.37,0.37,0.37}{#1}}
\newcommand{\OtherTok}[1]{\textcolor[rgb]{0.00,0.23,0.31}{#1}}
\newcommand{\PreprocessorTok}[1]{\textcolor[rgb]{0.68,0.00,0.00}{#1}}
\newcommand{\RegionMarkerTok}[1]{\textcolor[rgb]{0.00,0.23,0.31}{#1}}
\newcommand{\SpecialCharTok}[1]{\textcolor[rgb]{0.37,0.37,0.37}{#1}}
\newcommand{\SpecialStringTok}[1]{\textcolor[rgb]{0.13,0.47,0.30}{#1}}
\newcommand{\StringTok}[1]{\textcolor[rgb]{0.13,0.47,0.30}{#1}}
\newcommand{\VariableTok}[1]{\textcolor[rgb]{0.07,0.07,0.07}{#1}}
\newcommand{\VerbatimStringTok}[1]{\textcolor[rgb]{0.13,0.47,0.30}{#1}}
\newcommand{\WarningTok}[1]{\textcolor[rgb]{0.37,0.37,0.37}{\textit{#1}}}

\usepackage{longtable,booktabs,array}
\usepackage{calc} % for calculating minipage widths
% Correct order of tables after \paragraph or \subparagraph
\usepackage{etoolbox}
\makeatletter
\patchcmd\longtable{\par}{\if@noskipsec\mbox{}\fi\par}{}{}
\makeatother
% Allow footnotes in longtable head/foot
\IfFileExists{footnotehyper.sty}{\usepackage{footnotehyper}}{\usepackage{footnote}}
\makesavenoteenv{longtable}
\usepackage{graphicx}
\makeatletter
\newsavebox\pandoc@box
\newcommand*\pandocbounded[1]{% scales image to fit in text height/width
  \sbox\pandoc@box{#1}%
  \Gscale@div\@tempa{\textheight}{\dimexpr\ht\pandoc@box+\dp\pandoc@box\relax}%
  \Gscale@div\@tempb{\linewidth}{\wd\pandoc@box}%
  \ifdim\@tempb\p@<\@tempa\p@\let\@tempa\@tempb\fi% select the smaller of both
  \ifdim\@tempa\p@<\p@\scalebox{\@tempa}{\usebox\pandoc@box}%
  \else\usebox{\pandoc@box}%
  \fi%
}
% Set default figure placement to htbp
\def\fps@figure{htbp}
\makeatother





\setlength{\emergencystretch}{3em} % prevent overfull lines

\providecommand{\tightlist}{%
  \setlength{\itemsep}{0pt}\setlength{\parskip}{0pt}}



 
\usepackage[style=template/jpa,]{biblatex}
\addbibresource{bibliography.bib}


% フォントサイズと行間の設定
\usepackage{setspace}
\renewcommand{\normalsize}{\fontsize{12pt}{16.5pt}\selectfont}
\doublespacing

% ページ番号を下部中央に
\usepackage{fancyhdr}
\pagestyle{plain}
\fancypagestyle{plain}{
  \fancyhf{} 
  \fancyfoot[C]{\thepage} 
  \renewcommand{\headrulewidth}{0pt}}

% ランニングヘッドを無効化
\pagestyle{plain}

% ページサイズとジオメトリの設定
\usepackage{geometry}
\geometry{
  a4paper,
  left=4.1cm,
  right=4.1cm,
  top=3.0cm,
  bottom=3.0cm,
  footskip=2.0cm
}

% セクションの書式設定
\usepackage{titlesec}
% sectionを中央揃えに
\titleformat{\section}
  {\normalfont\Large\bfseries\centering}  % 中央揃え
  {\thesection}
  {1em}
  {}
% subsectionを左揃えに(デフォルトのまま)
\titleformat{\subsection}
  {\normalfont\large\bfseries}  % 左揃え
  {\thesubsection}
  {1em}
  {}
% subsubsectionも左揃えに(デフォルトのまま)
\titleformat{\subsubsection}
  {\normalfont\normalsize\bfseries}  % 左揃え
  {\thesubsubsection}
  {1em}
  {}
\usepackage{booktabs}
\usepackage{longtable}
\usepackage{array}
\usepackage{multirow}
\usepackage{wrapfig}
\usepackage{float}
\usepackage{colortbl}
\usepackage{pdflscape}
\usepackage{tabu}
\usepackage{threeparttable}
\usepackage{threeparttablex}
\usepackage[normalem]{ulem}
\usepackage{makecell}
\usepackage{xcolor}
\makeatletter
\@ifpackageloaded{caption}{}{\usepackage{caption}}
\AtBeginDocument{%
\ifdefined\contentsname
  \renewcommand*\contentsname{Table of contents}
\else
  \newcommand\contentsname{Table of contents}
\fi
\ifdefined\listfigurename
  \renewcommand*\listfigurename{List of Figures}
\else
  \newcommand\listfigurename{List of Figures}
\fi
\ifdefined\listtablename
  \renewcommand*\listtablename{List of Tables}
\else
  \newcommand\listtablename{List of Tables}
\fi
\ifdefined\figurename
  \renewcommand*\figurename{Figure}
\else
  \newcommand\figurename{Figure}
\fi
\ifdefined\tablename
  \renewcommand*\tablename{Table}
\else
  \newcommand\tablename{Table}
\fi
}
\@ifpackageloaded{float}{}{\usepackage{float}}
\floatstyle{ruled}
\@ifundefined{c@chapter}{\newfloat{codelisting}{h}{lop}}{\newfloat{codelisting}{h}{lop}[chapter]}
\floatname{codelisting}{Listing}
\newcommand*\listoflistings{\listof{codelisting}{List of Listings}}
\makeatother
\makeatletter
\makeatother
\makeatletter
\@ifpackageloaded{caption}{}{\usepackage{caption}}
\@ifpackageloaded{subcaption}{}{\usepackage{subcaption}}
\makeatother
\usepackage{bookmark}
\IfFileExists{xurl.sty}{\usepackage{xurl}}{} % add URL line breaks if available
\urlstyle{same}
\hypersetup{
  pdftitle={タイトル:はじめてのチュウ},
  pdfauthor={学籍番号:56 氏名:コロ助},
  colorlinks=true,
  linkcolor={blue},
  filecolor={Maroon},
  citecolor={Blue},
  urlcolor={Blue},
  pdfcreator={LaTeX via pandoc}}


\title{タイトル:はじめてのチュウ}
\author{学籍番号:56 氏名:コロ助}
\date{}
\begin{document}
\maketitle


\pagenumbering{gobble}
\setcounter{tocdepth}{2}
\renewcommand{\contentsname}{目次}
\tableofcontents
\newpage
\pagenumbering{arabic}
\setcounter{page}{1}

\section{序文}\label{ux5e8fux6587}

\subsection{はじめに}\label{ux306fux3058ux3081ux306b}

まず,\textcite{Abrams2020}
のように,すると,bibファイル内のAbramsの2020年の論文が引用されます。そして,次のように,{[}{]}でくくると文末の引用スタイルになります\autocite{Allport1935}。また,文末に複数引用する場合は,こういう感じにします\autocite{Bergson2002,Freud1956}。このQmdファイルではBibLatex-jpaを使っていますので,日本語文献も処理できます。例えば,\textcite{向田2009}
, \textcite{堀2009}, \textcite{矢嶋2013}
は,XXXについて示した\autocite{Helmholtz1925,Freud1956}などの文章も処理できます。

\clearpage

01234567890123456789012345678901234567890123456789012345678901234567890123456789012345678901234567890123456789012345678901234567890123456789012345678901234567890123456789012345678901234567890123456789012345678901234567890123456789012345678901234567890123456789012345678901234567890123456789012345678901234567890123456789012345678901234567890123456789012345678901234567890123456789012345678901234567890123456789012345678901234567890123456789012345678901234567890123456789012345678901234567890123456789012345678901234567890123456789012345678901234567890123456789012345678901234567890123456789012345678901234567890123456789012345678901234567890123456789012345678901234567890123456789012345678901234567890123456789012345678901234567890123456789012345678901234567890123456789012345678901234567890123456789
ここから八百字超えています。

\subsection{先行研究について}\label{ux5148ux884cux7814ux7a76ux306bux3064ux3044ux3066}

\subsubsection{先行研究での知見1}\label{ux5148ux884cux7814ux7a76ux3067ux306eux77e5ux898buxff11}

\subsubsection{先行研究での知見2}\label{ux5148ux884cux7814ux7a76ux3067ux306eux77e5ux898b2}

\subsubsection{先行研究での知見3}\label{ux5148ux884cux7814ux7a76ux3067ux306eux77e5ux898b3}

\subsubsection{先行研究での知見4}\label{ux5148ux884cux7814ux7a76ux3067ux306eux77e5ux898b4}

\subsubsection{先行研究での知見5}\label{ux5148ux884cux7814ux7a76ux3067ux306eux77e5ux898b5}

\subsubsection{先行研究での知見6}\label{ux5148ux884cux7814ux7a76ux3067ux306eux77e5ux898buxff16}

\subsubsection{先行研究での知見7}\label{ux5148ux884cux7814ux7a76ux3067ux306eux77e5ux898buxff17}

\subsection{先行研究の問題点}\label{ux5148ux884cux7814ux7a76ux306eux554fux984cux70b9}

\section{目的}\label{ux76eeux7684}

\clearpage

\section{方法}\label{ux65b9ux6cd5}

\subsection{研究参加者}\label{ux7814ux7a76ux53c2ux52a0ux8005}

神奈川県内の私立大学生2800名(男性919名,女性1881名)が実験or調査に参加した。参加者の平均年齢
(標準偏差) は,28.78歳(11.13)であった。

\subsection{行動課題 AND/OR 質問紙}\label{ux884cux52d5ux8ab2ux984c-andor-ux8ceaux554fux7d19}

\subsection{実験手続き OR 調査手続き}\label{ux5b9fux9a13ux624bux7d9aux304d-or-ux8abfux67fbux624bux7d9aux304d}

\begin{figure}[H]

{\centering \pandocbounded{\includegraphics[keepaspectratio]{figures/fig1.png}}

}

\caption{ミュラー・リヤー錯視の例}

\end{figure}%

\begin{figure}[H]
\centering
\includegraphics[clip,width = 8cm]{figures/fig1.png}
\caption{ミュラー・リヤー錯視の例}    
\end{figure}

\subsection{統計解析}\label{ux7d71ux8a08ux89e3ux6790}

 統計解析は,macOS Sequoia 15.0上で,R version 4.5.1
(2025-06-13)を用いて実施された。

\clearpage

\section{結果}\label{ux7d50ux679c}

\subsection{記述統計}\label{ux8a18ux8ff0ux7d71ux8a08}

\begin{table}

\caption{\label{tab:unnamed-chunk-3}Big Five因子の記述統計量}
\centering
\begin{tabular}[t]{lrrrrrrrr}
\toprule
  & n & Mean & SD & Median & Min & Max & Skewness & kurtosis\\
\midrule
Extraversion & 2713 & 18.96 & 2.71 & 19 & 5 & 29 & 0.01 & 1.08\\
Neuroticism & 2694 & 15.82 & 5.97 & 15 & 5 & 30 & 0.22 & -0.66\\
Conscientiousness & 2707 & 19.04 & 2.77 & 19 & 5 & 30 & -0.17 & 0.81\\
Agreeableness & 2709 & 21.04 & 3.68 & 22 & 5 & 30 & -0.66 & 0.68\\
Openness & 2726 & 19.34 & 2.74 & 19 & 5 & 29 & -0.02 & 1.09\\
\bottomrule
\multicolumn{9}{l}{\textsuperscript{} Note. SD=standard deviation}\\
\end{tabular}
\end{table}

\subsection{メインの解析の前提となる解析}\label{ux30e1ux30a4ux30f3ux306eux89e3ux6790ux306eux524dux63d0ux3068ux306aux308bux89e3ux6790}

\subsubsection{変数間の相関係数}\label{ux5909ux6570ux9593ux306eux76f8ux95a2ux4fc2ux6570}

\renewcommand{\arraystretch}{1}
\begin{landscape}\begin{table}

\caption{\label{tab:unnamed-chunk-4}Big Five因子の平均・標準偏差と相関}
\fontsize{10}{12}\selectfont
\begin{threeparttable}
\begin{tabular}[t]{lrrccccc}
\toprule
Variable & $N$ & $M$ & $SD$ & 1 & 2 & 3 & 4\\
\midrule
1. Extraversion & 2713 & 18.96 & 2.71 &  &  &  & \\
 &  &  &  &  &  &  & \\
 &  &  &  &  &  &  & \\
 &  &  &  &  &  &  & \\
2. Neuroticism & 2694 & 15.82 & 5.97 & .04$^{*}$ &  &  & \\
 &  &  &  & {}[.00, .08] &  &  & \\
 &  &  &  & $p$ = .049 &  &  & \\
 &  &  &  &  &  &  & \\
3. Conscientiousness & 2707 & 19.04 & 2.77 & .18$^{**}$ & .25$^{**}$ &  & \\
 &  &  &  & {}[.15, .22] & {}[.21, .28] &  & \\
 &  &  &  & $p$ < .001 & $p$ < .001 &  & \\
 &  &  &  &  &  &  & \\
4. Agreeableness & 2709 & 21.04 & 3.68 & .30$^{**}$ & -.14$^{**}$ & .06$^{**}$ & \\
 &  &  &  & {}[.26, .33] & {}[-.18, -.10] & {}[.02, .10] & \\
 &  &  &  & $p$ < .001 & $p$ < .001 & $p$ = .002 & \\
 &  &  &  &  &  &  & \\
5. Openness & 2726 & 19.34 & 2.74 & .25$^{**}$ & .16$^{**}$ & .25$^{**}$ & .19$^{**}$\\
 &  &  &  & {}[.22, .29] & {}[.12, .20] & {}[.21, .28] & {}[.15, .23]\\
 &  &  &  & $p$ < .001 & $p$ < .001 & $p$ < .001 & $p$ < .001\\
 &  &  &  &  &  &  & \\
\bottomrule
\end{tabular}
\begin{tablenotes}
\item \textit{Note}. \textit{N} = number of cases. \textit{M} = mean. \textit{SD} = standard deviation. Square brackets = 95\% confidence interval. \newline  * indicates $p$ < .05. ** indicates $p$ < .01.
\end{tablenotes}
\end{threeparttable}
\end{table}
\end{landscape}
\renewcommand{\arraystretch}{1}

\subsubsection{ヒストグラム}\label{ux30d2ux30b9ux30c8ux30b0ux30e9ux30e0}

\begin{figure}[H]

{\centering \pandocbounded{\includegraphics[keepaspectratio]{paper_files/figure-pdf/unnamed-chunk-5-1.pdf}}

}

\caption{\label{fig:figs}神経症傾向のヒストグラム}

\end{figure}%

\subsubsection{2群の比較(連続変数)}\label{ux7fa4ux306eux6bd4ux8f03ux9023ux7d9aux5909ux6570}

神経症傾向に関して性差を検討したところ,男性(\emph{M} = 14.74, \emph{SD}
= 5.72)よりも、女性(\emph{M} = 16.35, \emph{SD} =
6.03)の方が有意に神経症傾向が高かった(\emph{t} (1853.20) = 6.77,
\emph{p} = 0.00 , \emph{d} = ,95 \%CI {[}0.19, 0.35{]})。

\begin{figure}[H]

{\centering \pandocbounded{\includegraphics[keepaspectratio]{paper_files/figure-pdf/figs-1.pdf}}

}

\caption{\label{fig:figs}神経症傾向の平均と}

\end{figure}%

\subsubsection{2群の比較(離散変数)}\label{ux7fa4ux306eux6bd4ux8f03ux96e2ux6563ux5909ux6570}

\begin{table}

\caption{\label{tab:unnamed-chunk-7}性別と教育歴についてのクロス集計表}
\centering
\begin{tabular}[t]{lrr}
\toprule
\multicolumn{1}{c}{ } & \multicolumn{2}{c}{Education} \\
\cmidrule(l{3pt}r{3pt}){2-3}
  & High & Low\\
\midrule
\addlinespace[0.3em]
\multicolumn{3}{l}{\textbf{Gender}}\\
\hspace{1em}Female & 131 & 1608\\
\hspace{1em}Male & 93 & 745\\
\bottomrule
\end{tabular}
\end{table}

女性より、男性の方が高学歴者が多いことが示唆された( \(\chi ^2\) (1.00,
\emph{N} = 2800) = 8.61, \emph{p} = 0.00)。

\subsection{メインの解析の記載}\label{ux30e1ux30a4ux30f3ux306eux89e3ux6790ux306eux8a18ux8f09}

\subsubsection{重回帰分析}\label{ux91cdux56deux5e30ux5206ux6790}

\renewcommand{\arraystretch}{1}
\begin{landscape}\begin{table}

\caption{\label{tab:unnamed-chunk-9}重回帰分析結果}
\fontsize{10}{12}\selectfont
\begin{threeparttable}
\begin{tabular}[t]{lrrcrcrcc}
\toprule
Predictor & $b$ & 95\% CI & $beta$ & 95\% CI & Unique $R^2$ & 95\% CI & $r$ & Fit\\
\midrule
(Intercept) & 41.12** & {}[22.72, 59.53] &  &  &  &  &  & \\
adverts & 0.09** & {}[0.07, 0.10] & 0.52 & {}[0.44, 0.61] & .27** & {}[.18, .36] & .58** & \\
airplay & 3.59** & {}[3.02, 4.15] & 0.55 & {}[0.46, 0.63] & .29** & {}[.20, .38] & .60** & \\
 &  &  &  &  &  &  &  & $R^2$ = .629**\\
 &  &  &  &  &  &  &  & 95\% CI[.55,.69]\\
 &  &  &  &  &  &  &  & \\
\bottomrule
\end{tabular}
\begin{tablenotes}
\item \textit{Note}. $N$ = 200. $b$ = unstandardized regression weight. $beta$ = standardized regression weight. Unique $R^2$ = semipartial correlation squared. $r$ = zero-order correlation. CI = confidence interval. \newline  * indicates $p$ < .05. ** indicates $p$ < .01.
\end{tablenotes}
\end{threeparttable}
\end{table}
\end{landscape}
\renewcommand{\arraystretch}{1}

\subsubsection{分散分析の結果}\label{ux5206ux6563ux5206ux6790ux306eux7d50ux679c}

\renewcommand{\arraystretch}{1}\begin{table}

\caption{\label{tab:unnamed-chunk-10}分散分析結果}
\fontsize{10}{12}\selectfont
\begin{threeparttable}
\begin{tabular}[t]{lrrrrrrc}
\toprule
Predictor & $SS$ & $df$ & $MS$ & $F$ & $p$ & $\eta_{partial}^2$ & 90\% CI\\
\midrule
(Intercept) & 29403.12 & 1 & 29403.12 & 354.10 & <.001 &  & \\
gender & 156.25 & 1 & 156.25 & 1.88 & .177 & .04 & {}[.00, .17]\\
alcohol & 102.08 & 2 & 51.04 & 0.61 & .546 & .03 & {}[.00, .12]\\
gender x alcohol & 1978.12 & 2 & 989.06 & 11.91 & <.001 & .36 & {}[.15, .49]\\
Error & 3487.50 & 42 & 83.04 &  &  &  & \\
\bottomrule
\end{tabular}
\begin{tablenotes}
\item \textit{Note}. $SS$ = Sum of squares. $df$ = degrees of freedom. $MS$ = mean square. CI indicates the confidence interval for $\eta_{partial}^2$.
\end{tablenotes}
\end{threeparttable}
\end{table}
\renewcommand{\arraystretch}{1}

\subsection{メインの解析結果を補強する解析の記載}\label{ux30e1ux30a4ux30f3ux306eux89e3ux6790ux7d50ux679cux3092ux88dcux5f37ux3059ux308bux89e3ux6790ux306eux8a18ux8f09}

\clearpage

\section{考察}\label{ux8003ux5bdf}

\subsection{主要な発見の概要}\label{ux4e3bux8981ux306aux767aux898bux306eux6982ux8981}

\subsection{考えられるメカニズムの考察と説明}\label{ux8003ux3048ux3089ux308cux308bux30e1ux30abux30cbux30baux30e0ux306eux8003ux5bdfux3068ux8aacux660e}

\subsection{関連のある先行研究の結果との比較}\label{ux95a2ux9023ux306eux3042ux308bux5148ux884cux7814ux7a76ux306eux7d50ux679cux3068ux306eux6bd4ux8f03}

\subsection{研究結果が与える示唆}\label{ux7814ux7a76ux7d50ux679cux304cux4e0eux3048ux308bux793aux5506}

\subsection{研究の限界と今後の課題}\label{ux7814ux7a76ux306eux9650ux754cux3068ux4ecaux5f8cux306eux8ab2ux984c}

\subsection{結論}\label{ux7d50ux8ad6}

\clearpage

\section{要約}\label{ux8981ux7d04}

\clearpage

\section{引用文献}\label{ux5f15ux7528ux6587ux732e}

\printbibliography[heading=none]

\clearpage

\section{謝辞}\label{ux8b1dux8f9e}

\clearpage

\section{付録}\label{ux4ed8ux9332}

\begin{Shaded}
\begin{Highlighting}[]
\FunctionTok{library}\NormalTok{(tidyverse)}
\end{Highlighting}
\end{Shaded}






\end{document}
